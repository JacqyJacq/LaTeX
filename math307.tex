\documentclass[11pt]{exam}
\RequirePackage{amssymb, amsfonts, amsmath, latexsym, verbatim, xspace, setspace, mathrsfs}
\usepackage{amsmath,amsthm,amssymb,amsfonts, 
hyperref, color, graphicx}
\RequirePackage{tikz, pgflibraryplotmarks}
\usepackage[margin=1in]{geometry}
\usepackage{graphicx}

\newcommand{\N}{\mathbb{N}}
\newcommand{\Z}{\mathbb{Z}}
\newcommand{\Q}{\mathbb{Q}}
 
\newenvironment{problem}[2][Problem:]{\begin{trivlist}
\item[\hskip \labelsep {\bfseries #1}\hskip \labelsep {\bfseries #2}]}{\end{trivlist}}

\newenvironment{claim}[2][Claim:]{\begin{trivlist}
\item[\hskip \labelsep {\bfseries #1}\hskip \labelsep {\bfseries #2}]}{\end{trivlist}}

\newenvironment{defn}[2][Definition:]{\begin{trivlist}
\item[\hskip \labelsep {\bfseries #1}\hskip \labelsep {\bfseries #2}]}{\end{trivlist}}

% Here's where you edit the Class, Exam, Date, etc.
\newcommand{\class}{Math 307}
\newcommand{\term}{Spring 2016}
\newcommand{\examnum}{Suggested Exam 1}
\newcommand{\examdate}{Due: 3/3/16}
\newcommand{\timelimit}{now until then}

% For an exam, single spacing is most appropriate
\singlespacing
% \onehalfspacing
% \doublespacing

% For an exam, we generally want to turn off paragraph indentation
\parindent 0ex

\begin{document} 

% These commands set up the running header on the top of the exam pages
\pagestyle{head}
\firstpageheader{}{}{}
\runningheader{\class}{\examnum\ - Page \thepage\ of \numpages}{\examdate}
\runningheadrule

\begin{flushright}
\begin{tabular}{p{2.8in} r l}
\textbf{\class} & \textbf{Name:} & \makebox[2in]{\hrulefill}\\
\textbf{\term} &&\textbf{\examnum}\\
\textbf{\examdate} &&
\textbf{Time Limit:  \timelimit}  \\ 
\end{tabular}\\
\end{flushright}
\rule[1ex]{\textwidth}{.1pt}




\begin{minipage}[t]{3.7in}
\vspace{0pt}
\begin{itemize}

\item \textbf{DO NOT open the exam booklet until you are told to begin. You should write your name and section number at the top and read the instructions.}

\vfill

\item Organize your work, in a reasonably neat and coherent way, in
the space provided. If you wish for something to not be graded, please strike it out neatly. I will grade only work on the exam paper, unless you clearly indicate your desire for me to grade work on additional pages.

\item You may use any results from class, homework or the text, but you must cite the result you are using. You must prove everything else.

\item You needn't spend your time rewriting definitions or axioms on the exam.

\end{itemize}


\end{minipage}
\hfill
\begin{minipage}[t]{2.3in}
\vspace{0pt}
%\cellwidth{3em}
\gradetablestretch{2}
%Uncomment this line to make the table display 100 as the total no matter what. This is good for tests with an ommit question.
%\settabletotalpoints{100}
\vqword{Problem}
\addpoints % required here by exam.cls, even though questions haven't started yet.	
\gradetable[v]%[pages]  % Use [pages] to have grading table by page instead of question

\end{minipage}

\begin{itemize}


\item You may use the text, my class notes and/or any notes and study guides you have created. You may use a calculator. You may not use a cell phone or computer.


\item When you have completed your test, hand it to me and go have a great weekend!

\item There is a single bonus problem at the end of the test. It would be best to work first on the main test as this problem is only worth 5 points and will be graded strictly.

\end{itemize}

\newpage 
\begin{questions}
\addpoints

\question[10] This is a 10 point question. I can change the number of points in the square brackets and it will add up automatically for me.

\question This is a question with parts. I can set the amount of points for each part and it will add them together to put on the front page.
\begin{parts}
\part[5] part one
\vfill
\part[8] another part
\vfill
\part[5] I put vfills in between questions to fill the space. They will automatically take up all the space and divide it evenly between all the vfills. Don't forget to put on at the end of the page!
\vfill
\end{parts}


\newpage 
\addpoints

\question[18] Every time you start a new page, you need to tell it to add points. I don't know why.
\vfill

\bonusquestion[5] This is a bonus question. It has points but they are not added on the cover page.
\vfill

\end{questions}
\end{document}
